\chapter{Template}

Normal text \dots

Another paragraph \dots

\section{Second level heading}

Normal text \dots

\subsection{Third level heading} % strongly dicouraged

This is a normal text paragraph. This is a complementary sentence. This is a complementary sentence. Here is a footnote.\footnote{Footnotes are not recommended, but if used, look like this.} Citations should be in Harvard style. For example: \citet{markus.1988} is used often, for information systems development \citep{ward_benefits_2012} is used more commonly. \citet[p.~275]{kautz.1994} is an important paper, too.

This is a normal text paragraph. This is a complementary sentence. This is a complementary sentence. Here is a footnote.\footnote{Footnotes are not recommended, but if used, look like this.} Citations should be in Harvard style. For example: \citet[p.~275]{ward_managing_2007} is used often, for information systems development \citep{avison.1995} is used more commonly. \citet[p.~275]{kautz.1994} is an important paper, too.


\begin{figure}[H]
  \ECISfigureexample % ? Remove this example.
  % \includegraphics[width=\linewidth]{your_fancy_graphics_file} % ? Use this for external files.
  \caption{This is a figure caption. The caption should always be placed below the figure.}
\end{figure}

\noindent % ? Use \noindent if you don't want a new paragraph with indentation.
Tables are inserted as shown in the example below.

\begin{table}[H]
  \begin{tabular}{@{}|l|c|c|@{}}
    \hline
    \bigstrut[t] Question                  & Average 1992 & Average 1999\\
    \hline
    \bigstrut[t] 1 How do you regard \dots & 3.4 & 3.7\\
                 2 How do you \dots        & 2.7 & 3.4\\
                 3 How do you \dots        & 3.9 & 3.6\\
    \hline
  \end{tabular}
  \caption{This is a table caption. The caption should always be placed below the table. Here's some text to show how a long caption runs longer than a single line.}
\end{table}

\noindent
You can use lists, both unnumbered and numbered. They can also be nested. Here is an unnumbered list:

\begin{itemize}
\item This is an item in the first level list.
\item This is another item.
  \begin{itemize}
  \item Here we are in the second level.
  \item This is another item.
  \end{itemize}
\item And a third item on the first level.
\end{itemize}

Here is a numbered list:

\begin{enumerate}
\item This is an item in the first level list.
\item This is another item.
  \begin{enumerate}
  \item Here we are in the second level.
  \item This is another item.
  \end{enumerate}
\item And a third item on the first level.
\end{enumerate}


Informationssysteme - Probleme bei der Realisierung langfristiger Benefits
Anwendung des Benefits Management in den 90er Jahren
Projekterfolge durch Einführung von Informationsyssteme
Aufkommen verschiedener Arten von Informationensystemen
Trennung von transaktionalen und analytischen Systeme
Einführung von BI-Systemen (nach wie vor sehr teuer, langfristiger Nutzen schwierig zu messen)
Neue Herausforderungen für analytischen System mit Big Data
Sicherung des Erfolgs eines Unternehmens durch Einsatz von Business Intelligence
Konzentration auf zwei Aspekte:
Anwendung des Benefits Management auch auf BI-Systeme
Steigerung der Benefits Management Akzeptanz durch Verwendung von Design Prinzipien für BI-Systeme
Anwendung von Design Prinzipien für BI-Systeme zur höheren