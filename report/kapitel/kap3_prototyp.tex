% !TEX root = ../Gruppenbildung.tex
\chapter{Prototyp}
\label{prototyp}

In diesem Kapitel wird zunächst die Architektur und die verwendeten Technologien näher eingegangen. Anschliessend wird anhand verschiedener Screenshots die Bedienung der Webanwendung erläutert.

\section{Architektur und Technologien}
\label{architektur_technologie}

Der \emph{groupfindr} Prototyp demonstriert die technische Machbarkeit einer skalierbaren real-time Anwendung mit \emph{socket.io}. Er veranschaulicht die Darstellungs- und Interaktionsmöglichkeiten der interaktiven Graphikbibliothek \emph{EaselJS} und verwendet einen Standardstack an Webtechnologien zur Umsetzung einer responsiven Webanwendung.
\newline\newline
Die event-basierte Javascriptbibliothek socket.io\footnote{\url{http://socket.io/}} ermöglicht die bidirektionale Kommunikation zwischen Client und Server in Echtzeit und bildet die Basis für die \emph{groupfindr} Funktionalität Bewegungen anderer Benutzer in Echtzeit zu verfolgen. Bewegt ein Benutzer seinen Avatar, wird seine neue Position (\code{pos}) über einen etablierten Websocket (\code{socket}) vom Client-Browser an den Server gesendet: \code{socket.emit('updatepos', pos);}. Der Server reagiert auf den \code{updatepos} Event mit dem in Listing \ref{lst:updatepos} gezeigten Callback. Bei der Positionsüberprüfung wird festgestellt ob der Benutzer durch die Positionsänderung soeben eine Gruppe verlassen oder betreten hat. Anschließend wird die neue Position (\code{newPos}) an alle Benutzer im selben Raum gesendet.

\begin{lstlisting}[caption=Server Implementation des \emph{updatepos} Event, label=lst:updatepos]
socket.on('updatepos', function (pos) { 
  // ... 
  // Broadcast new position to all other player in the same room 
  app.io.to(pos.roomname).emit('update', newPos); 
  // ... 
}
\end{lstlisting}

Eine Herausforderung der Echzeitkommunikation dieser Art besteht in der Sicherstellung einer konsistenten Sicht über alle verbundenen Benutzer hinweg. Durch die häufigen parallelen Updates können leicht Logikfehler im Event-Handling entstehen, die beispielsweise dazu führten, dass einzelne Benutzer dupliziert wurden. Des Weiteren hat die Terminologie von Räumen (\emph{rooms}) für etwas Verwirrung gesorgt, da dieser Begriff sowohl in der technischen (als socket.io Namensraum) als auch in der konzeptionellen (als virtuellen Raum in dem sich Gruppen bilden) Domäne existiert.
\newline\newline
Die in CreateJS enthaltene Javascriptbibliothek EaselJS wird zur dynamischen Darstellung des Raum Canvas verwendet, insbesondere auch zur Animation des beweglichen Benutzer-Avatars. Dafür bietet EaselJS\footnote{\url{http://www.createjs.com/easeljs}} einen eleganten Weg über sogenannte SpriteSheets, siehe Abbildung~\ref{fig:spritesheet}. Das Codesnippet in Listing~\ref{lst:spritesheet} zeigt wie diese einzelnen Bilder oder eine Abfolge von Bildern für Bewegungen in bestimmte Richtungen mit einer konfigurierbaren Geschwindigkeit (\code{0.4}) definiert werden können. Eine nennenswerte Herausforderung war die korrekte relative Berechnung der Positionskoordinaten beim Skalieren des Raum Canvas.

\begin{figure}[htbp]
\begin{tabular}{p{0.45\textwidth}p{0.5\textwidth}}
    \begin{minipage}{.5\textwidth}
    \centering
    \includegraphics[width=0.45\linewidth]{graphiken/player.png}
    \caption{Avatar SpriteSheet}
    \label{fig:spritesheet}
    \end{minipage}
    &
    \begin{minipage}{.5\textwidth}
\begin{lstlisting}[caption=EaselJS Avatar Animationen,label=lst:spritesheet]
createjs.SpriteSheet({
  "images": [image],
  frames: {width: 64, height: 100, regX: 32, regY: 50, count: 16},
  animations: {
    "standdown": 0,
    "standleft": 4,
    "standright": 8,
    "standup": 12,
    "down": [1, 3, "standdown", 0.4],
    "left": [5, 7, "standleft", 0.4],
    "right": [9, 11, "standright", 0.4],
    "up": [13, 15, "standup", 0.4]
  }
});
\end{lstlisting}
\end{minipage}
\end{tabular}
\end{figure}

\emph{groupfindr} verwendet einen Standardstack an Webtechnologien zur Umsetzung einer responsiven Webanwendung. Serverseitig kommt Node.js\footnote{\url{https://nodejs.org/}}, Express\footnote{\url{http://expressjs.com/}}, und die Template Engine Jade\footnote{\url{http://jade-lang.com/}} zum Einsatz. Die Websocket Verbingungen zwischen Client und Server werden durch socket.io abstrahiert. Clientseitig verwenden wir Bootstrap\footnote{\url{http://getbootstrap.com/}}, CSS, und Javascript mit den Bibliotheken EaselJS, jsPDF\footnote{\url{http://parall.ax/products/jspdf}} und JQuery\footnote{\url{https://jquery.com/}}.
\newline\newline
Die ausklappbare Chatbox am unteren Bildrand (siehe Abbildung~\ref{groupfindr_einfaerbung-chat}) kann als Einfallstor für Injection Attacken auf alle Benutzer im gleichen Raum dienen. Chatnachrichten erlauben die Verwendung von HTML Tags (z.B. \code{<b>html</b>}). Während solch harmlose Formatierungen kein Problem darstellen, lässt sich durch die Verwendung eines HTML Skript-Tags beliebiger Code im Browser aller Benutzer ausführen wie das Beispiel in Abbildung \ref{groupfindr_einfaerbung-chat} zeigt. Hierbei leitet das Senden der Chatnachricht \code{<script>window.location = "https://google.com";</script>} alle Benutzer im gleichen Raum unmittelbar auf die Google Website weiter. Sicherheitskritische Attacken dieser Art lassen sich verhindern, indem der Server jede Chatnachricht überprüft und vor der Weiterleitung an andere Benutzer alle HTML Elemente escaped.

\section{How To}
\label{how_to}

Auf der Startseite der \emph{groupfindr} Webanwendung\footnote{\url{https://groupfindr.herokuapp.com/}} kann der Benutzer Benutzername und Raum auswählen (siehe Abbildung~\ref{fig:login_screen}). Existierende Räume werden ihm dabei als Dropdown-Auswahl zur Verfügung gestellt. Sobald ein Benutzer einen virtuellen Raum (dargestellt als rechteckige begehbare Fläche) betritt, erhält er einen Avatar zugewiesen (siehe Abbildung~\ref{fig:room_joined}). Im virtuellen Raum kann der Benutzer seinen Avatar über Pfeiltasten, per Maus, oder touchfreundlich per Drag’n’Drop bewegen. Dabei kann jeder Benutzer die Laufwege aller Benutzer im gleichen Raum in Echtzeit mitverfolgen.

\begin{figure}
\centering
\begin{tabular}{cc}
\subfloat[Login-Screen]{\includegraphics[width=0.3\linewidth]{graphiken/login.png}\label{fig:login_screen}} & 
\multirow{-7}[9.3]{*}{\subfloat[Raum beigetreten]{\includegraphics[width=0.6\linewidth]{graphiken/joined-canvas.png}\label{fig:room_joined}}} \\
\subfloat[Gruppe erstellen]{\includegraphics[width=0.3\linewidth]{graphiken/create-group.png}\label{fig:group_create}} &
\end{tabular}
\caption{\emph{groupfindr} - Einstieg \& Gruppenerstellung (Quelle: Eigene Darstellung)}
\label{groupfindr_einstieg-gruppenerstellung}
\end{figure}

\begin{figure}[h]
\centering
\includegraphics[width=\linewidth]{graphiken/groups-and-chat.png}
\caption{\emph{groupfindr} - Einfärbung der Gruppen und Chatbox (Quelle: Eigene Darstellung)}
\label{groupfindr_einfaerbung-chat}
\end{figure}

Solange noch freie Plätze auf dem Canvas bestehen (in diesem Prototyp aus Darstellungsgründen auf 7 begrenzt), kann jeder Benutzer über den blauen Gruppenbutton im Canvas eine neue Gruppe mit Namen und Beschreibung erstellen (siehe Abbildung~\ref{fig:group_create}). Jeder Benutzer kann einer Gruppe betreten indem er seinen Avatar in das Feld der Gruppe seiner Wahl bewegt (siehe Abbildung~\ref{groupfindr_einfaerbung-chat}). Gruppen mit optimaler Mitgliederanzahl (4-6) werden dabei grün eingefärbt, solche mit zu vielen Mitgliedern sind an der roten Einfärbung schnell zu erkennen. Jeder Benutzer kann leere Gruppen wieder löschen um Platz für weitere Gruppen zu schaffen. Die aktuellen Gruppeneinteilungen sind jederzeit in einer textuellen Übersicht, links neben dem Canvas einsehbar. Diese Übersicht lässt sich am Ende des Gruppenfindungsprozesses über die Exportfunktion (\emph{Export groups}) als PDF herunterladen.